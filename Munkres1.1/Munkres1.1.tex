\documentclass[]{article}
\usepackage{amsmath,amssymb}
\usepackage{lmodern}
\usepackage{iftex}
\usepackage{ragged2e}
\usepackage{stackengine}
\usepackage{dirtytalk}
\usepackage{graphicx}
% \graphicspath{{./images/}}


\hbadness=99999

\date{October 2022}
\author{ID: 14221}
\title{Munkres 1.1}

\begin{document}

\maketitle

\begin{enumerate}
    \item  Check the distributive laws for $\cup$ and $\cap$ and DeMorgan's laws.
    \newline Consider A, B, and C to be sets. 
    \newline a. 
    \newline Proof that $A \cap (B \cup C)=(A\cap B)\cup(A\cap C)$.
    \\\\$\{x|x \in A\cap(B\cup C)\} \iff \{x|x\in A \land x\in (B \cup C)\}$
    \newline $\{x|x\in A \land x\in (B \cup C)\} \iff \{x|x \in A \land (x \in B \lor x \in C)\}$
    \newline $\{x|x \in A \land (x \in B \lor x \in C)\} \iff \{x|(x\in A \land x \in B)\lor(x\in A \land x \in C)\}$
    \newline $\{x|(x\in A \land x \in B)\lor(x\in A \land x \in C)\} \iff \{x|x \in A \cap B \lor x \in A \cap C\}$
    \newline $\{x|x \in A \cap B \lor x \in A \cap C\} \iff \{x|x \in (A \cap B) \cup (A \cap C)\}$
    \\\\b.
    \newline Proof that $A \cup (B \cap C)=(A \cup B)\cap(A \cup C)$.
    \\\\$\{x|x \in A \cup (B \cap C)\} \iff \{x|x \in A \lor x \in (B \cap C)\}$
    \newline $\{x|x \in A \lor x \in (B \cap C)\} \iff \{x|x \in A \lor (x \in B \land x \in C)\}$
    \newline $\{x|x \in A \lor (x \in B \land x \in C)\} \iff \{x|(x \in A \lor x \in B) \land (x \in A \lor x \in C)\}$
    \newline $\{x|(x \in A \lor x \in B) \land (x \in A \lor x \in C)\} \iff \{x|x \in A \cup B \land x \in A \cup C\}$
    \newline $\{x|x \in (A \cup B) \cap (A \cup C)\}$
    \\\\c.
    \newline First proof for DeMorgan's law using $A - (B \cup C) = (A - B) \cap (A - C)$.
    \\\\$\{x|x \in A - (B \cup C)\} \iff \{x|x \in A \land x \notin (B \cup C)\}$
    \newline $\{x|x \in A \land x \notin (B \cup C)\} \iff \{x|x \in A \land \neg(x \in B \lor x \in C)\}$
    \newline $\{x|x \in A \land \neg(x \in B \lor x \in C)\} \iff \{x|x \in A \land (x \notin B \land x \notin C)\}$
    \newline $\{x|x \in A \land (x \notin B \land x \notin C)\} \iff \{x|(x \in A \land x \notin B) \land (x \in A \land x \notin C)\}$
    \newline $\{x|(x \in A \land x \notin B) \land (x \in A \land x \notin C)\} \iff \{x|x \in (A - B) \land x \in (A - C)\}$
    \newline $\{x|x \in (A - B) \land x \in (A - C)\} \iff \{x|x \in (A - B) \cap (A - C)\}$
    \\\\d.
    \newline Second proof for DeMorgan's law using $A - (B \cap C) = (A - B) \cup (A - C)$
    \\\\$\{x|x \in A - (B \cap C)\} \iff \{x|x \in A \land x \notin B \cap C\}$
    \newline $\{x|x \in A \land x \notin B \cap C\} \iff \{x|A \land \neg (x \in B \land x \in C)\}$
    \newline $\{x|A \land \neg (x \in B \land x \in C)\} \iff \{x|x \in A \land (x \notin B \lor x \notin C)\}$
    \newline $\{x|x \in A \land (x \notin B \lor x \notin C)\} \iff \{x|(x \in A \land x \notin B) \lor (x \in A \land x \notin C)\}$
    \newline $\{x|(x \in A \land x \notin B) \lor (x \in A \land x \notin C)\} \iff \{x|x \in (A - B) \lor x \in (A - C)\}$
    \newline $\{x|x \in (A - B) \lor (A - C)\} \iff \{x|x \in (A - B) \cup (A - C)\}$
    
    \item a. $A \subset B$ and $A \subset C \iff A \subset (B \cup C)$
    \\\\ The double implication fails and $\longrightarrow$ is true and $\longleftarrow$ is false.
    \newline Proof for $\longrightarrow$:
    \newline Using $A \subset B$ and $A \subset C$, consider $x \in A$. Then, $x \in B$ since $A \subset B$ so that $x \in B \cup C$. Therefore, $A \subset (B \cup C)$ is true.
    \\\\ Proof for $\longleftarrow$:
    \\However, for the converse, consider A = {0,1,2}, B={0,1}, and C={2,3}. Then $A \subset {0,1,2,3} = B \cup C$  but it is false that $A \subset B$ since 2 is in A but not in B and $A \subset C$ is false since 0 is in A but not in C. Thus,$
    A \subset B and  A \subset C \leftarrow A \subset (B \cup  C) $ does not hold.
    \\\\e. $A-(A-B)=B$
    \\\\$A-(A-B) \subset B$ is true but $A-(A-B) \supset B$ is false.
    \newline Using $x \in A-(A-B)$ and $x \in A$, then $x \notin A-B$. Therefore it is false that $x \in A$ and $x \notin B$. So $x \notin A$ or $x \in B$ is true. So, if $x \in A$, $x \in B$. Thus $A-(A-B) \subset B$. \\However, for $A-(A-B) \supset B$, let $A={0,1}$ and $B={1,2}$. Therefore, $A-B={0}$ and $A-(A-B)={1}$. This means that B is not a subset of $A-(A-B)$ since $2 \in B$ but $2 \notin A-(A-B)$.
    \\Thus, $A-(A-B)=B$ is false.
    \\\\i. $(A \cap B) \cup (A-B)=A$
    \\\\Proof that $(A \cap B) \cup (A-B)=A$ is true.
    \newline $(A \cap B) \cup (A-B)$
    \newline $=\{x|(x \in A \land x \in B) \lor (x \in A \land \neg (x \in B))\}$
   \newline $=\{x|(x \in A \land x \in B \lor x \in A) \land (x \in A \land x \in B \lor \neg (x \in B ))\}$ Distr
   \newline $=\{x|(x \in A \land x \in A \}$ Simp
   \newline $=\{x|(x \in A \}$ Rep
    \newline $=A $
    \\\\o. $A \times (B-C)=(A \times B)-(A \times C)$
    \\\\Proof that $A \times (B-C)=(A \times B)-(A \times C)$ is true.
    \newline $A \times (B-C)$
    \newline $=\{(x,y)|x \in A \land y\in B \land \neg (y\in C)\}$
    \newline $=\{(x,y)|x \in A \land y\in B \land \neg (y\in C) \lor \neg (x \in A)\}$ Add
    \newline $=\{(x,y)|x \in A \land y\in B \land \neg (y\in C \land x \in A)\}$ DeM
    \newline $=(A \times B)-(A \times C)$ 
    \newline $Thus,A \times (B-C)=(A \times B)-(A \times C)$ is true.
    \item a.  Write the contrapositive and converse of the following statement: \say{If $x<0$, then $x^2-x>0$,} and determine which (if any) of the three statements are true.
    \\\\ Contrapositive: If $x^2-x \leq 0$ then $x \geq 0$
    \newline Converse: If $x^2-x>0$ then $x<0$
    \\\\ Original: The original statement is true. Since $x^2-x$ can be rewritten as $x(x-1)$ and $x$ is negative, $x^2-x$ will be positive as the product of two negative numbers is positive.
    \\\\ Contrapositive: Since the original statement is true, the contrapositive will also be true since it is logically equivalent to the original statement.
    \\\\ Converse: The converse statement is false. For example, consider the case where x=2. $2^2-2>0$ but $2<0$ is false thus proving that the converse is false.
    \\\\b. Write the contrapositive and converse of the following statement: \say{If $x>0$, then $x^2-x>0$,} and determine which (if any) of the three statements are true.
    \\\\ Contrapositive: If $x^2-x \leq 0$ then $x \leq 0$
    \newline Converse: If $x^2-x>0$ then $x>0$
    \\\\Original: The original statement is false. For example, consider the case where x=0.2. $0.2$ is positive but $0.2^2-0.2=-0.16$ which is negative. Therefore, the original statement is false.
    \\\\Contrapositive: Since the contrapositive is logically equivalent to the original statement, the contrapositive will also be false.
    \\\\Converse: The converse is also false. For example, consider the case where $x=-2$. $-2^2-2>0$ but $-2>0$ is false and thus the converse is false.
    
    \item Let $A$ and $B$ be sets of real numbers. Write the negation of each of the following statements:
    \\\\ a. For every $a \in A$, it is true that $a^2 \in B$.
    \newline Negation: There is at least one $a \in A$ where $a^2 \notin B$.
    \\\\ b. For at least one $a \in A$, it is true that $a^2 \in B$.
    \newline Negation: For every $a \in A$, $a^2 \notin B$.
    \\\\ c. For every $a \in A$, it is true that $a^2 \notin B$.
    \newline Negation: There is at least one $a \in A$ where $a^2 \in B$.
    \\\\ d. For at least one $a \notin A$, it is true that $a^2 \in B$.
    \newline Negation: For every $a \notin A$, $a^2 \notin B$.
    
    \item Let $\alpha$ be a nonempty collection of sets. Determine the truth of each of the following statements and of their converses:
    \\\\ a. $x \in \bigcup_{A \in \alpha} A \Rightarrow x \in A$ for at least one $A \in \alpha$.
    \newline The original statement and its converse are true since the statement on the right is the definition of the statement on the left.
    \\\\b. $x \in \bigcup_{A \in \alpha} A \Rightarrow x \in A$ for every $A \in \alpha$.
    \newline The original statement is false. For example, consider the case where $\alpha =\{\{0\},\{1\}\}$. Then $\bigcup_{A \in \alpha} A = \{0,1\}$ so that $0 \in \cup_{A \in \alpha}A$, but 0 is not in $\alpha$ for every $A \in \alpha$ since $0 \notin \{1\}$
    \\\\ The converse is true. Consider the case where $x \in A$ for every $A \in \alpha$. Since $\alpha$ is nonempty, $A_0 \in \alpha$. Then $x \in A_0$ since $A_0 \in \alpha$. Therefore, $x \in \bigcup_{A \in \alpha} A$ is true since $x \in A_0$ and $A_0 \in \alpha$.
    \\\\c. $x \in \bigcap_{A \in \alpha} \Rightarrow x \in A$ for at least one $A \in \alpha$.
    \newline The original statement is true. From the definition of $x \in \bigcap_{A \in \alpha}$, $x \in A$ for every $A \in \alpha$. Since $\alpha$ is nonempty, there is an $A_0 \in \alpha$ so that $x \in A_0$. Therefore, $A_0 \in \alpha$.
    \\\\ The converse is false.
    \newline Consider the case where $\alpha = \{\{0,1\},\{1,2\}\}$. Then $0 \in \{0,1\}$ and $\{0,1\} \in \alpha$, but $1 \notin \bigcap_{A \in \alpha} \alpha$.
    \\\\d. $x \in \bigcap_{A \in \alpha} A \Rightarrow x \in A$ for at least one $A \in \alpha$.
    \newline The original statement and its converse are true since the statement on the right is the definition of the statement on the left.
    
    \item Write the contrapositive of each of the statements of Exercise 5.
    \\\\ a. $x \notin A$ for every $A \in \alpha \Rightarrow x \notin \bigcup_{A \in \alpha}A$
    \\\\ b. $x \notin A$ for at least one $A \in \alpha \Rightarrow x \notin \bigcup_{A \in \alpha}A$
    \\\\ c. $x \notin A$ for every $A \in \alpha \Rightarrow x \notin \bigcap_{A \in \alpha}A$
    \\\\ d. $x \notin A$ for at least one $A \in \alpha \Rightarrow x \notin \bigcap_{A \in 
    \alpha}A$
    
    \item Given sets $A$, $B$, and $C$, express each of the following sets in terms of $A$, $B$, and $C$, using the symbols $\cup$, $\cap$, and $-$.
    \\\\ $D=A \cap (B \cup C)$
    \newline $E=(A \cap B) \cup C$
    \\\\ For F:
    \newline $x \in F \iff x \in A \land (x \in B \Rightarrow x \in C)$
    \newline $A \land (x \in B \Rightarrow x \in C) \iff A \land (x \notin B \lor x \in C)$
    \newline $A \land (x \notin B \lor x \in C) \iff x \in A \land \neg(x \in B \land x \notin C)$
    \newline $x \in A \land \neg(x \in B \land x \notin C) \iff x \in A \land \neg (x \in B - C)$
    \newline $x \in A \land \neg (x \in B - C) \iff x \in A \land x \notin B - C$
    \newline $x \in A \land x \notin B - C \iff x \in A - (B-C)$
    \newline $F = A-(B-C)$
    
    \item If a set $A$ has two elements, show that $\mathcal{P}(A)$ has four elements. How many elements does $\mathcal{P}(A)$ have if $A$ has one element? Three elements? No elements? Why is $\mathcal{P}(A)$ called the power set of $A$?
    \\\\a. If set $A$ has two elements: $A = \{0,1\}$. $\mathcal{P}(A) = \{\{ \emptyset \},\{0\},\{1\},\{0,1\}\}$ and $\mathcal{P}(A)$ has 4 elements. $A=2$, $\mathcal{P}(A)=2^2=4$.
    \\\\b. If set $A$ has one element: $A = \{0\}$. $\mathcal{P}(A) = \{\{ \emptyset \},\{0\}\}$ and $\mathcal{P}(A)$ has 2 elements. 
    
    
    
    
    
    
    
    
    
    
    $A=1$, $\mathcal{P}(A)=2^1=2$.
    \\\\c. If set $A$ has three elements: $A = \{0,1,2\}$. $\mathcal{P}(A) = \{\{ \emptyset \},\{ 0 \},\{ 1 \},\{ 2 \}, \{ 0,1 \}, \{ 0,2 \}, \{ 1,2 \}, \{ 0,1,2 \}\}$ and $\mathcal{P}(A)$ has 3 elements. $A=3$, $\mathcal{P}(A)=2^3=8$.
    \\\\d. If set $A$ has zero elements: $A = \{ \emptyset \}$. $\mathcal{P}(A) = \{\{ \emptyset \}\}$ and $\mathcal{P}(A)$ has 1 elements. $A=0$, $\mathcal{P}(A)=2^0=1$.
    \\\\e. $\mathcal{P}(A)$ is called the power set of $A$ since $\mathcal{P}(A)=2^{A}$. So the number of elements in $\mathcal{P}(A)$ is 2 to the power of number of elements in $A$.
    
    \item Formulate and prove DeMorgan's laws for arbitrary unions and intersections.
    \\ Given that $A$ is a set and $C$ is a nonempty collection of sets.
    \\For arbitrary unions:
    \\ $A - \bigcup_{B \in C}B = \bigcap_{B \in C} (A-B)$
    \\\\Proof:
    \\$A - \bigcup_{B \in C}B \iff x \in A \land \neg \exists B \in C(x \in B)$
    \\$x \in A \land \neg \exists B \in C(x \in B) \iff x \in A \land \forall B \in C(x \notin B)$
    \\$x \in A \land \forall B \in C(x \notin B) 
    \iff \forall B \in C(x \in A \land x \notin B)$
    \\$\forall B \in C(x \in A \land x \notin B) \iff \forall B \in C(x \in A - B)$
    \\$\forall B \in C(x \in A-B) \iff x \in \bigcap_{B \in C}(A-B)$
    \\\\For arbitrary intersections:
    \\ $A - \bigcap_{B \in C}B = \bigcup_{B \in C} (A-B)$
    \\\\Proof:
    \\$A - \bigcap_{B \in C}B \iff x \in A \land x \notin \bigcap_{B \in C} B$
    \\$x \in A \land x \notin \bigcap_{B \in C} B \iff x \in A \land \neg \forall B \in C(x \in B)$
    \\$x \in A \land \neg \forall B \in C(x \in B) \iff x \in A \land \exists B \in C (x \notin B)$
    \\$x \in A \land \exists B \in C (x \notin B) \iff \exists B \in C(x \in A - B)$
    \\$\exists B \in C(x \in A - B) \iff x \in \bigcup_{B \in C}(A-B)$
    
    \item Let $\mathbb{R}$ denote the set of real numbers. For each of the following subsets of $\mathbb{R} \times \mathbb{R}$, determine whether it is equal to the Cartesian product of two subsets of $\mathbb{R}$.
    \\\\a. $\{(x,y)|x$ is an integer$\}$.
    \\ Equal to $\mathbb{Z} \times \mathbb{R}$
    \\\\b. $\{(x,y)|0<y \leq 1 \}$.
    \\ Equal to $\mathbb{R} \times (0,1]$. $(a,b]$ means $\{x \in \mathbb{R}|a<x \leq b\}$.
    \\\\c. $\{(x,y)|y>x\}$.
    \\Not equal to Cartesian product of subsets of $\mathbb{R}$
    \\Suppose that $A = \{ (x,y) | y>x \}$ and $A=B \times C$ where $B, C \in R$. Since $1>0$, $(0,1) \in A$. Then $0 \in B$ and $1 \in C$ since $A = B \times C$. We also have that $1 \in B$ and $1 \in C$ so $(1,1) \in B \times C=A$ but it is false that $1>1$ so there is a contradiction and $A$ isn't $B \times C$.
    \\\\d. $\{(x,y)|x$ is an integer and $y$ is an integer $\}$.
    \\ This is equal to $(\mathbb{R}-\mathbb{Z}) \times \mathbb{Z}$
    \\ The first part proves x cannot be an integer as it is part of a set of real numbers with integer removed. The second part is self-evident.
    \\\\e. $\{(x,y)|x^2+y^2<1\}$.
    \\Not equal to Cartesian product of subsets of $\mathbb{R}$.
        \\Suppose that $A = \{ (x,y) | x^2+y^2<1 \}$ and $A=B \times C$ where $B,C \in \mathbb{R}$. For example, $(8/10)^2+0^2=64/100+0=64/100<1$ so that $(8/10,0) \in A = B \times C$, and thus $8/10 \in B$ and $0 \in C$. Also $0^2 + (8/10)^2 = 64/100<1$ so that $(0, 8/10) \in A=B \times C$, and $0 \in B$ and $8/10 \in C$. Thus $(8/10,8/10) \in B \times C = A$ since $8/10$ is in $B$ and $C$. Since $(8/10)^2+(8/10)^2=64/100 + 64/100 = 128/100 \geq 1$, $(8/10,8/10)$ cannot be in $A$ so there is a contradiction. Therefore, A isn't equal to $B \times C$.
\end{enumerate}




\end{document}
